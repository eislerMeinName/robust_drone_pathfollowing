\begin{figure}[htp]
\begin{python}
	#imports
	...
	
	if __name__ == "__main__":
		parser = argparse.ArgumentParser(description="Script that allows to train your RL Model")
		parseparameters(parser)
		ARGS = parser.parse_args()
		check(**vars(ARGS))
		...
		
		# Create training environment 
		sa_env_kwargs: dict = dict(aggregate_phy_steps=5, obs=ObservationType('kin'), act=act, mode=mode,
		total_force=total_force, upper_bound=upper_bound, drone_model=drone, gui=gui,
		debug=debug_env, episode_len=episode_len)
		train_env = make_vec_env(WindSingleAgentAviary, env_kwargs=sa_env_kwargs, n_envs=cpu, seed=0)
	
		onpolicy_kwargs: dict = dict(activation_fn=torch.nn.ReLU, net_arch=[256, 256])
		if load == DEFAULT_LOAD:
			model = PPO(a2cppoMlpPolicy,
				train_env,
				policy_kwargs=onpolicy_kwargs,
				tensorboard_log='results/tb/',
				verbose=1)
		else:
			model = PPO.load(load, train_env, tensorboard_log='results/tb/')
		# Create evaluation Environment 
		eval_env = make_vec_env(WindSingleAgentAviary, env_kwargs=sa_env_kwargs, n_envs=cpu, seed=0)
		# Train the model 
		... # Callbacks...
		model.learn(total_timesteps=steps, callback=eval_callback, 
		log_interval=100)
		# Save the model 
		model.save(name)
\end{python}
\caption{shortened excerpt of the python implementation of the learning script}
\end{figure}